\documentclass[12pt]{article}
\usepackage{amsmath}
\usepackage{setspace}
\usepackage{indentfirst}
\newcommand{\numpy}{{\tt numpy}}    % tt font for numpy

\topmargin -.5in
\textheight 9in
\oddsidemargin -.25in
\evensidemargin -.25in
\textwidth 7in
\begin{document}

% ========== Edit your name here
\author{Chengze(Kol) Zuo}
\title{DS-GA 1008 Homework Assignment 1}
\date{February 14, 2019}
\maketitle

\medskip

% ========== Begin answering questions here

\section*{Part I}
\subsection*{\indent1.1}

According to the instructions, let \textbf{x} be the column vector
$\left[
\begin{matrix}
x_{1}\\
x_{2}
\end{matrix}
\right]$, let \textbf{y} be the column vector
$\left[
\begin{matrix}
y_{1}\\
y_{2}
\end{matrix}
\right]$, and let the weight matrix \textbf{W} be the 2x2 matrix
$\left[
\begin{matrix}
w_{1,1} & w_{1,2}\\
x_{2,1} & w_{2,2}
\end{matrix}
\right]$. Applying the $\textbf{y} = \textbf{W}\textbf{x} + \textbf{b}$ can give us $\textbf{y} = 
\left[\begin{matrix}
w_{1,1}x_{1}+w_{1,2}x_{2}+b_{1}\\
w_{2,1}x_{1}+w_{2,2}x_{2}+b_{2}
\end{matrix}\right]$. For the $\frac{\partial L}{\partial\textbf{W}}$, its shape would be same as the \textbf{W}, and $\frac{\partial L}{\partial\textbf{W}} = 
\left[\begin{matrix}
\frac{\partial L}{\partial w_{1,1}} & \frac{\partial L}{\partial w_{1,2}}\\
\frac{\partial L}{\partial w_{2,1}} & \frac{\partial L}{\partial w_{2,2}}
\end{matrix}\right]$. For now, we focus on solving $\frac{\partial L}{\partial w_{1,1}}$, and then apply the same approach to $\frac{\partial L}{\partial w_{1,2}}$, $\frac{\partial L}{\partial w_{2,1}}$, etc.\setstretch{1.5}According to the chain rule, we can infer that:
\begin{center}
    $\frac{\partial L}{\partial w_{1,1}} = \frac{\partial L}{\partial \textbf{y}}\cdot\frac{\partial\textbf{y}}{\partial w_{1,1}}$
\end{center}, where:
\begin{center}
    $
    \frac{\partial L}{\partial \textbf{y}} = 
    \left[\begin{matrix}
    \frac{\partial L}{\partial y_{1}}\\
    \frac{\partial L}{\partial y_{2}}
    \end{matrix}\right] = 
    \left[\begin{matrix}
    dy_{1}\\
    dy_{2}
    \end{matrix}\right]$
\end{center}, and $\frac{\partial\textbf{y}}{\partial w_{1,1}}$ equals to:
\begin{center}
    $
    \frac{\partial\textbf{y}}{\partial w_{1,1}}=
    \left[\begin{matrix}
    \frac{\partial y_{1}}{\partial w_{1,1}}\\
    \frac{\partial y_{2}}{\partial w_{1,1}}
    \end{matrix}\right]=
    \left[\begin{matrix}
    x_{1}\\
    0
    \end{matrix}\right]
    $
\end{center}Therefore, after applying the dot product between two vectors, we can get:
\begin{center}
    $
    \frac{\partial  L}{\partial w_{1,1}} = 
    dy_{1}x_{1}
    $
\end{center}Therefore, we can infer that:
\begin{center}
    $\frac{\partial L}{\partial\textbf{W}} = 
    \left[\begin{matrix}
    \frac{\partial L}{\partial w_{1,1}} & \frac{\partial L}{\partial w_{1,2}}\\
    \frac{\partial L}{\partial w_{2,1}} & \frac{\partial L}{\partial w_{2,2}}
    \end{matrix}\right]=
    \left[\begin{matrix}
    dy_{1}x_{1} & dy_{1}x_{2}\\
    dy_{2}x_{1} & dy_{2}x_{2}
    \end{matrix}\right]=
    \left[\begin{matrix}
    dy_{1}\\
    dy_{2}
    \end{matrix}\right]
    \left[\begin{matrix}
    x_{1} & x_{2}
    \end{matrix}\right]=
    \frac{\partial L}{\partial\textbf{y}}\mathbf{x}^T
    $
\end{center}Using the same approach, for the $\frac{\partial L}{\partial \mathbf{b}}$, we can infer that:
\begin{center}
    $\frac{\partial L}{\partial \mathbf{b}} = \frac{\partial L}{\partial \textbf{y}}\cdot\frac{\partial\mathbf{y}}{\partial \mathbf{b}}$
\end{center}where:
\begin{center}
    $
    \frac{\partial\mathbf{L}}{\partial \mathbf{b}}=
    \left[\begin{matrix}
    \frac{\partial\mathbf{L}}{\partial b_{1}} \\
    \frac{\partial\mathbf{L}}{\partial b_{2}} 
    \end{matrix}\right]
    $
\end{center}    
\begin{center}    
    $
    \frac{\partial\textbf{y}}{\partial b_{1}}=
    \left[\begin{matrix}
    \frac{\partial y_{1}}{\partial b_{1}} \\
    \frac{\partial y_{2}}{\partial b_{1}} 
    \end{matrix}\right]=
    \left[\begin{matrix}
    1\\
    0
    \end{matrix}\right]
    $
\end{center}Therefore, $\frac{\partial L}{\partial b_{1}}$ would be:
\begin{center}
    $
    \frac{\partial L}{\partial b_{1}}=
    \left[\begin{matrix}
    dy_{1} \\
    dy_{2}
    \end{matrix}\right]\cdot\left[\begin{matrix}
    1\\
    0
    \end{matrix}\right] = dy_{1}
    $
\end{center}Also, $\frac{\partial L}{\partial b_{2}}$ would be $dy_{2}$. Thus: 
\begin{center}
    $
    \frac{\partial L}{\partial \mathbf{b}}= 
    \left[\begin{matrix}
    dy_{1} \\
    dy_{2}
    \end{matrix}\right]=
    \frac{\partial\mathbf{L}}{\partial\mathbf{y}}
    $
    \begin{center}
        (i.e., $\mathbf{I}\frac{\partial\mathbf{L}}{\partial\mathbf{y}}$, the Jacobian matrix $\frac{\partial\mathbf{y}}{\partial\mathbf{b}}$ is an Identity matrix)
    \end{center}
\end{center}



\subsection*{\indent1.2}

\begin{equation*}
    a^2 + b^2 = c^2.
\end{equation*}

Some people like to write scalars without boldface $x+y=1$ and vectors or matrices in boldface
\begin{equation}
    \mathbf{A} \mathbf{x} = \mathbf{b}.
\end{equation}

An example of a matrix \LaTeX:
\begin{equation}
    \mathbf{A} = \left(
    \begin{array}{ccc}
    3 & -1 & 2 \\ 	
    0 & 1 & 2 \\ 
    1 & 0 & -1 \\
\end{array} 
\right).  
\end{equation}

\end{document}
\grid
\grid